\documentclass[english,a4paper,nologo]{europasscv}
\ecvname{Weipeng Yao}
\ecvaddress{CHAMBRE 201, 61 BD JOURDAN, 75014, PARIS, FRANCE}
\ecvtelephone{+33 06 43 65 42 49}
\ecvemail{yao.weipeng@polytechnique.edu, weipeng.yao@obspm.fr}
\ecvdateofbirth{- May 22, 1990}
\ecvnationality{- Chinese}
\ecvgender{- Male}
\ecvpicture[width=1.5in]{Photo_WeipengYAO}
\ecvpictureleft
\usepackage{hyperref}

\definecolor{csc}{rgb}{0.0,0.0,1}

\hypersetup{
colorlinks,
urlcolor=csc,
    pdfauthor={Weipeng Yao},
    pdftitle={Weipeng Yao - Curriculum Vitae},
    pdfsubject={Curriculum Vitae},
    pdfkeywords={Curriculum Vitae, CV, Weipeng Yao, HEDLA}
}

\begin{document}
	\begin{europasscv}
	


	
	\newpage
	\ecvpersonalinfo

 	\ecvsection{Education}  
		\ecvtitle{\color{black} 2015.9 -- 2019.7}{\color{black} \small Ph. D.: Plasma Physics, School of Physics, Peking University, Beijing, China} 			
		\ecvtitle{\color{black} 2012.9 -- 2015.7}{ \color{black} \small Master of Science: Plasma Physics, Graduate School of China Academy of Engineering Physics, Beijing, China}
  		\ecvtitle{\color{black} 2008.9 -- 2012.7}{ \color{black} \small Bachelor of Science: Physics ({\it National Training Bases for Talented Students in Fundamental Sciences}), School of Physics , Shanxi University, Taiyuan, China}


\ecvsection{Research experience}
	\ecvtitle{\color{black} 2019.10 -- 2021.9}{\color{black} \small LULI and LERMA, Sorbonne University, UPMC Univ. Paris 06}
		\ecvitem{Post-doc Researcher}{Advisor: \href{https://portail.polytechnique.edu/luli/en/dr-julien-fuchs}{Prof. Julien Fuchs} \& \href{https://sites.google.com/site/andreaciardihomepage/home}{Prof. Andrea Ciardi} }
		\ecvitem{}{\rm \small Focuses on numerical modeling of laboratory and astrophysical plasma systems under magnetic fields.}
%		\ecvitem{Thesis}{Kinetic study of relativistic jet and plasmas interaction in high energy astrophysics}

	\ecvtitle{\color{black} 2015.9 -- 2019.7}{\color{black} \small Center for Applied Physics and Technology, Peking University, Beijing, China}
		\ecvitem{Ph.D. Candidate}{Advisor: Prof. Xian-tu He \& \href{http://capt.pku.edu.cn/kydw/yjry/zxcy/822209.htm}{Prof. Bin Qiao} }
		\ecvitem{}{\rm \small Focuses on the kinetic simulation of astrophysical relativistic jet transport in plasma, including the competing of different kinetic instabilities, magnetic reconnection, particle acceleration via collisionless shock and $\gamma$-ray radiation via quantum electrodynamics (QED) effects.}
		\ecvitem{Thesis}{Kinetic study of relativistic jet and plasmas interaction in high energy astrophysics}

	\ecvtitle{\color{black} 2017.12}{\color{black} \small Shanghai Institute of Laser Plasma, China Academy of Engineering Physics, Shanghai, China}
		\ecvitem{Experimental participant}{Advisor: Prof. Wenbin Pei}
		\ecvitem{}{\rm \small Focuses on the asymmetrical laser-driven magnetic reconnection (MR) using proton radiography at SG-II-upgrade laser facility.}

	\ecvtitle{\color{black} 2012.9 -- 2015.7}{\color{black} \small Graduate School of China Academy of Engineering Physics, Beijing, China}
		\ecvitem{Master Candidate}{Advisor: Prof. Baiwen Li}
		\ecvitem{}{\rm \small Focuses on the generation of high-quality proton beam based on laser plasma interaction, including stabilization via multi-species target and injection  via plasma density modulation.}
		\ecvitem{Thesis}{Particle simulation research on monochromatic proton acceleration via ultra-short ultra-intense laser pulse and multi-component plasma interaction}
		
		
\ecvsection{Recent Research Interests}
%	\ecvtitle{\color{black} }{\color{blue} \small Under extreme conditions, i.e. in high energy astrophysical phenomena -- }
%		\ecvitem{Relativistic Jet}{the kinetic effects of their basic parameters, i.e. the component, velocity, jet-to-ambient density ratio and magnetic field around, when transport in ambient plasmas}
%%		\ecvitem{}{\it the kinetic effects of their basic parameters, i.e. the component, velocity, jet-to-ambient density ratio and magnetic field around}
%		\ecvitem{Kinetic instabilities}{the excitation, competition and saturation of different kinetic instability modes}
%%		\ecvitem{}{\it the dispersion relation under different parameter space, especially when multi-dimensional effects are considered}
%		\ecvitem{Magnetic reconnection}{the onset, reconnection rate, the definition under relativistic regime and the QED effects}
%%		\ecvitem{}{\it As the kinetic energy of jet particles transfer to electromagnetic (EM) fields through instabilities or magnetic dynamos, magnetic reconnection (MR) can transfer the energy back from EM fields to particles. However, what is the onset of MR under different parameter space? How fast is the reconnection rate? And how to define MR under relativistic regime, when X- {\rm \&} O-points are moving with the observer?}
%		\ecvitem{Collisionless shock}{the formation time, Mach number, and duration}
%%		\ecvitem{}{\it When will the collisionless shock form after the dominated instability mode saturates? What determines its strength? And how long will it last? Will collisionless shock drive MR under certain conditions?}
%		\ecvitem{Particle acceleration}{the dominant mechanism and the resulting spectrum, as the potential origin of UHECRs}
%%		\ecvitem{}{\it Actually, all above processes lead to the energy exchange between different species. With some particles losing energy and some gaining, a particular energy spectrum is formed. Detailed kinetic study is necessary to find out the dominant mechanism in shaping the spectrum.}
%		\ecvitem{Radiation}{With further consideration of secondary QED effects, such as photon-photon pair creation, the radiation spectrum may occur some interesting features.}
%%		\ecvitem{}{\it After all, it is our ultimate goal to connect the radiation results with the observed one.}
%
%	\ecvtitle{\color{black} }{\color{blue} \small Under laboratory conditions, i.e. in high-energy-density regime produced by high-energy lasers}
%		\ecvitem{Non-relativistic Jet}{the collimation mechanism (the effect of radiative cooling, ambient plasma and magnetic field)}
%%		\ecvitem{}{\it the collimation mechanism (the effect of radiative cooling, ambient plasma and magnetic field)}
%		\ecvitem{MR in HED regime}{the dominated heating/acceleration/collision mechanism}
%		\ecvitem{Plasma instabilities}{fluid instabilities, e.g. kink mode, in the transport of non-relativistic flow in ambient plasma}
%		\ecvitem{Proton radiography}{diagnose the magnetic field structure in above experiments}
	
	\ecvitem{}{\color{black}  \rm With the development of laser technology, it will be more and more convenient to create plasmas under high-energy-density (HED) regime in laboratory. My research interest lies in this new, physically rich and exciting area at the intersection of traditional plasma physics, high-energy astrophysics and HED plasma physics. Specifically, I am interested in the astrophysical jet transport/collimation, plasma instabilities, collisonless/collisional shock formation, and particle acceleration, especially when magnetic fields are externally applied. In addition, I am also interested in advanced numerical modeling of related systems with both kinetic Particle-in-Cell codes and extended magnetohydrodynamic codes.}

%	\ecvitem{}{\color{blue}  \rm Astrophysical phenomena are rich in exciting physics and can provide us challenging opportunities to advance our understanding of the Universe. With the development of laser technology, it will be more convenient to create plasmas under HED regime in laboratory. My research interest lies in this new, physically rich and exciting area at the intersection of traditional plasma physics, high-energy astrophysics and high-energy-density plasma physics.}

% \ecvsection{Honours and Awards}
% 	\ecvitem{2018.9}{National Scholarship}
% 	\ecvitem{2017.12}{Second class of Collaborative Innovation Center of IFSA Scholarship}	
% 	\ecvitem{2017.10}{Best Poster Award of the 7th National Conference On HEDP, Xi'an, China}
% 	\ecvitem{2016.9}{Special Scholarship for PhD student, Peking University}		
% 	\ecvitem{2016.5}{Third Prize of the 2016 “Zhong Shengbiao Academic Forum”, Peking Univ., Beijing, China}	
% 	\ecvitem{2015.7}{Outstanding graduate, Graduate School of CAEP}	
% 	\ecvitem{2015.6}{Second class of Academic Scholarship, Graduate School of CAEP}	
% 	\ecvitem{2014.6}{Excellent Graduate Student Award, Graduate School of CAEP}	
% 	\ecvitem{2009-2011}{Undergraduate Scholarship, Shanxi University}
%
%
\ecvsection{Synergic Activities}
	\ecvitem{Invited lecture, 2021}{ Laser Plasma Physics Course, ELI}
	\ecvitem{Invited referee, 2020}{ New Astronomy}
	\ecvitem{Conference co-organizer, 2018}{The 4th International Conference on High Energy Density Physics, Ningbo, China}
	\ecvitem{Conference co-organizer, 2016}{The 3rd International Conference on High Energy Density Physics, Shenzhen, China}
	\ecvitem{Funding proposal, 2016}{the Special Program for Applied Research on Super Computation of the NSFC-Guangdong Joint Fund (the second phase)}
	\ecvitem{Invited referee, 2015}{ Laser and Particle Beams}
%	
%
\ecvsection{Technical Skills}
	\ecvtitle{\color{black} \it Language {\rm \&} Software}{}
		\ecvitem{  \rm Proficient }{ \rm  Python, \LaTeX, Linux/Unix, Adobe illustrator, Fortran}
		\ecvitem{  \rm Master }{  \rm Bash, C/C++, Matlab, VisIt, ParaView}
		\ecvitem{ \rm Familiar with }{ \rm  Vim, OpenMP/MPI}
				
	\ecvtitle{\color{black} \it Code Projects}{}
		\ecvitem{\rm 2019 - 2021 }{\small  \rm Integrated test particle module in 3D Ex-MHD code GORGON, written by Fortran.}
		\ecvitem{\rm 2018 - 2019 }{\small  \rm Familiar with the radiation hydrodynamic code MULTI.}
		\ecvitem{\rm 2018 - 2019 }{\small  \rm Familiar with the HEDP Simulation using FLASH Code.}
		\ecvitem{\rm 2017 - 2018 }{\small  \rm Integrated particle injection module in PIC code EPOCH, written by Fortran.}
		\ecvitem{\rm 2016 - 2017 }{\small  \rm Integrated QED photon merging functions in PIC code EPOCH, written by Fortran.}
		\ecvitem{\rm 2015 - 2016 }{\small  \rm Scripts for 3D data visualization and simulation results animation, written by Python/Matlab.}
		\ecvitem{\rm 2013 - 2014 }{\small  \rm Integrated multi-species module in PIC codes, written by Fortran.}
%	
%	
\ecvsection{Scientific Talks}
	\ecvitem{2021.8}{\small the 9th International Symposium ``Modern Problems of Laser Physics'' (MPLP-2021), Novosibirsk/Online, Russia}
	\ecvitem{\it Invited Oral}{\it Laboratory evidence for proton energization by collisionless shock surfing}	
	\ecvitem{2021.6}{\small the Satellite Meeting 2021 (EPS) of the 47th Conference on Plasma Physics, Online}
	\ecvitem{\it Oral}{\it Enhanced laser coupling, matter heating, and particle acceleration through Spatially-separated and Symmetrically-overlapped PW Lasers}
	\ecvitem{2021.4}{\small INTERNATIONAL CONFERENCE ON HIGH ENERGY DENSITY SCIENCES 2021, Osaka Univ./Online, Japan}
	\ecvitem{\it Invited Oral}{\it Laboratory evidence for proton energization by collisionless shock surfing}	
	\ecvitem{2018.10}{\small The 4th International Conference on High Energy Density Physics, Ningbo, China}
	\ecvitem{\it Oral}{\it Kinetic PIC simulations for transport of astrophysical relativistic jet in ambient plasmas}	
	\ecvitem{2018.4}{\small The 2018 ``Zhong Shenbiao'' Academic Forum, Beijing, China}
	\ecvitem{\it Oral}{\small {\it Kinetic effects of astrophysical relativistic flow transport in interstellar medium}}
%	\ecvitem{2017.10}{\small The 2017 APS Division of Plasma Physics Meeting, Milwaukee, WI, USA}
%	\ecvitem{\it Poster}{\small {\it Relay transport of relativistic flows in extreme magnetic fields of stars}}
	\ecvitem{2017.10}{\small The 2017 APS Division of Plasma Physics Meeting, Milwaukee, WI, USA}
	\ecvitem{\it Oral}{\small {\it Achieving stable radiation pressure acceleration of heavy ions via successive electron replenishment from ionization of a high-Z material coating}}
%	\ecvitem{2017.10}{\small The 7th National Conference On High Energy Density Physics, Xi'an, China}
%	\ecvitem{\it Poster}{{\it Relay transport of relativistic flows in extreme magnetic fields of stars}}	
%	\ecvitem{2017.3}{\small The 2017 High energy density physics young scientist forum, Shanghai, China}
%	\ecvitem{\it Poster}{{\it Shower effects on the transport of relativistic flow around stars}}		
%	\ecvitem{2016.9}{\small The 3rd International Conference on High Energy Density Physics, Shenzhen, China}
%	\ecvitem{\it Poster}{\it Shower effects on the propagation of relativistic jet around neutron star surface}	
	\ecvitem{2016.4}{\small The 2016 ``Zhong Shenbiao'' Academic Forum, Beijing, China}
	\ecvitem{\it Oral}{\it High quality proton beam generation by a combined mechanism using multi-component target}
	\ecvitem{2015.4}{\small The 2015 High energy density physics young scientist forum, Beijing, China}
	\ecvitem{\it Oral}{\it High quality proton beam generation by a combined mechanism using multi-component target}
	\ecvitem{2014.9}{\small The 2rd International Conference on High Energy Density Physics, Beijing, China}
	\ecvitem{\it Oral}{\it Generation of monoenergetic proton beams from a combined scheme with a CH target irradiated by ultraintense laser pulse}
%	\ecvitem{2014.4}{\small The 8th International West Lake Symposium on Laser Plasma Interaction, Hangzhou, China}
%	\ecvitem{\it Poster}{{\it Generation of monoenergetic proton beams from a combined scheme with a CH target irradiated by ultraintense laser pulse}}
%	\ecvitem{2013.9}{\small The first Symposium on the Frontier of High Energy Density Physics, Beijing, China}
%	\ecvitem{\it Poster}{\it Generation of monoenergetic proton beams from multi-component target irradiated by circularly polarized laser pulse}
%  		
%

\newpage

\ecvsection{First-author Research publications}

		% \ecvitem{ }{For an up to date and exhaustive list of articles see my profile on \href{https://scholar.google.com/citations?hl=en&user=gzxsWFIAAAAJ}{\underline{google scholar}}}

		\ecvitem{}{\bf For an up to date and exhaustive list of articles see my profile on {\href{https://scholar.google.com/citations?hl=en&user=gzxsWFIAAAAJ}{\underline{google scholar}}}}

		\ecvtitle{\color{black} {\it \small submit to Matter Radiat. at Extremes}}{\small \href{https://arxiv.org/abs/2105.13800}{\underline{Characterization of the stability and dynamics of a laser-produced plasma expanding}}}
		 \ecvitem{\rm \small (under review)} {\small \href{https://arxiv.org/abs/2105.13800}{\underline{across strong magnetic field}}}
		\ecvitem{\rm \small arXiv:2105.13800}{\rm {\bf W. Yao}, J. Capitaine, B. Khiar, T. Vinci, K. Burdonov, J. Béard, J. Fuchs, A. Ciardi}

		\ecvtitle{\color{black} {\it \small submit to Matter Radiat. at Extremes}}{\small \href{https://arxiv.org/abs/2104.12170}{\underline{Detailed characterization of laboratory magnetized super-critical collisionless shock and}}}
		 \ecvitem{\rm \small (under review)} {\small \href{https://arxiv.org/abs/2104.12170}{\underline{of the associated proton energization}}}
		\ecvitem{\rm \small arXiv:2104.12170}{\rm {\bf W. Yao}, A. Fazzini, S. N. Chen, K. Burdonov, P. Antici, J. Béard, S. Bolaños, A. Ciardi, R.Diab, E.D. Filippov, S. Kisyov, V. Lelasseux, M. Miceli, Q. Moreno, V. Nastasa, S. Orlando, S.Pikuz, D. C. Popescu, G. Revet, X. Ribeyre, E. d'Humières, J. Fuchs}

		\ecvtitle{\color{black} {\it \small Nat. Phys. (Online)}}{\small \href{https://www.nature.com/articles/s41567-021-01325-w}{\underline{Laboratory evidence for proton energization by collisionless shock surfing}}}
		\ecvitem{\rm \small (2021)}{\rm {\bf W. Yao}, A. Fazzini, S. N. Chen, K. Burdonov, P. Antici, J. Béard, S. Bolaños, A. Ciardi, R. Diab, E. D. Filippov, S. Kisyov, V. Lelasseux, M. Miceli, Q. Moreno, V. Nastasa, S. Orlando, S. Pikuz, D. C. Popescu, G. Revet, X. Ribeyre, E. d'Humières, J. Fuchs}

       	\ecvtitle{\color{black} {\it \small Astrophysical J.}, {\bf 876}, {\rm 2}}{\small \href{https://iopscience.iop.org/article/10.3847/1538-4357/ab13a0/meta}{\underline{Kinetic Particle-in-cell Simulations of the Transport of Astrophysical Relativistic Jets }}}
       	\ecvitem{\rm \small (2019)} {\small \href{https://iopscience.iop.org/article/10.3847/1538-4357/ab13a0/meta}{\underline{in Magnetized Intergalactic Medium}}}
		\ecvitem{}{\rm {\bf {W. P. Yao}}, B. Qiao, Z. Zhao, Z. Lei, H. Zhang, C. T. Zhou, S. P. Zhu and X. T. He}
%		\ecvitem{}{\rm This paper focuses on the kinetic effects of external applied perpendicular magnetic field on relativistic jet transport. We find that flatter spectrum energy distribution (SED) is achieved and the power law slope of the SED decreases as the magnetization rate increases.}

		\ecvtitle{\color{black} {\it \small New J. Physics}, {\bf 20}, {\rm 053060}}{\small \href{http://iopscience.iop.org/article/10.1088/1367-2630/aac5b8/meta}{\underline{The baryon loading effect on relativistic astrophysical jet transport in the interstellar medium}}}
		\ecvitem{\rm \small (2018)}{\rm {\bf W. P. Yao}, B. Qiao,  Z. Xu, H. Zhang, H. X. Chang, Z. H. Zhao, C. T. Zhou, S. P. Zhu and X. T. He }
%		\ecvitem{}{\rm We build a numerical platform to systematically study the baryon loading effect (BLE) on relativistic jet transport in the interstellar medium. Our results show that the BLE leads to a stronger collisionless shock formation and a reshape of phase space distribution.}
			
	\ecvtitle{\color{black} {\it \small Phys. Plasmas}, {\bf 24}, {\rm 082904}}{\small \href{http://aip.scitation.org/doi/10.1063/1.4996903}{\underline{Relay transport of relativistic flows in extreme magnetic fields of stars}}}
		\ecvitem{\rm \small (2017)}{\rm {\bf W. P. Yao}, B. Qiao,  Z. Xu, H. Zhang, H. X. Chang, C. T. Zhou, S. P. Zhu, X. G. Wang and X. T. He}
%		\ecvitem{\color{blue} \rm \small \it Featured {\rm \&} Cover Story}{\rm We build a relay transport model with the QED effects, including the $\gamma$-photon synchrotron radiation and magnetic pair creation processes. Our results, highlighted by \href{https://aip.scitation.org/doi/10.1063/1.4998692}{\underline{\rm AIP Scilight}}, are closely associated with the jet generation around the neutron star of X-Ray Binaries.}
		
	\ecvtitle{\color{black} {\it \small Phys. Plasmas}, {\bf 23}, {\rm 013107}}{\small \href{https://aip.scitation.org/doi/abs/10.1063/1.4940331}{\underline{Optimization of the combined proton acceleration regime with a target composition scheme}}}
		\ecvitem{\rm \small (2016)}{\rm {\bf W. P. Yao}, B. W. Li,  C. Y. Zheng, Z. J. Liu, X. Q. Yan and B. Qiao}
%		\ecvitem{}{\rm This paper focuses on stabilizing a long-distance combined proton acceleration by using multi-component target, where the protons get stably accelerated under multi-dimensional cases.}
						
    \ecvtitle{\color{black} {\it \small Laser Part. Beams}, {\bf 32}, {\rm 583-589}}{\small \href{https://doi.org/10.1017/S0263034614000561}{\underline{Generation of monoenergetic proton beams by a combined scheme with an overdense }}}
    	\ecvitem{\rm \small (2014)}{\small \href{https://doi.org/10.1017/S0263034614000561}{\underline{hydrocarbon target and an underdense plasma gas irradiated by ultra-intense laser pulse}}}
	\ecvitem{}{\rm {\bf W. P. Yao}, B. W. Li,  L. H. Cao, F. L. Zheng, T. W. Huang, C. Z. Xiao and M. M. Skoric}
%	\ecvitem{}{\rm This paper focuses on an optimized scheme using mixed hydrocarbon target, which helps reduce the requirement of the laser intensity to get protons to tens of GeV.}	
	
  %      \ecvtitle{\color{black} {\it \small Phys. Plasmas}, {\bf 24}, {\rm 092102}}{\small \href{http://dx.doi.org/10.1063/1.4997609}{\underline{Magnetic X points disturbed by the in-plane electric fields}}}
		% \ecvitem{\rm \small (2017)}{\rm Z. Xu, B. Qiao, {\bf W. P. Yao}, H. X. Chang, C. T. Zhou, S. P. Zhu and X. T. He}
%		\ecvitem{}{\rm This paper focuses on the definition of MR in relativistic regime, where the concept of the X and O points is ambiguous. My contribution is primarily on results analysis.}
	
  %       \ecvtitle{\color{black} {\it \small Plas. Phys. C. F.}, {\bf 59}, {\rm 064002}}{\small \href{https://doi.org/10.1088/1361-6587/aa6803}{\underline{Magnetic reconnection in the high-energy density regime}}}
		% \ecvitem{\rm \small (2017)}{\rm B. Qiao, {Z. Xu}, {\bf W. P. Yao}, H. X. Chang and X. T. He}
%		\ecvitem{}{\rm The paper follows the invited talk by my advisor Prof. Qiao. My contribution is primarily on the 3D PIC simulations of magnetic reconnection and the data processing \& visualization.}
			
  %    \ecvtitle{\color{black} {\it \small Phys. Plasmas}, {\bf 24}, {\rm 043111}}{\small \href{https://aip.scitation.org/doi/abs/10.1063/1.4981213}{\underline{Ultraintense laser absorption and $\gamma$-ray synchrotron radiation in near critical density plasmas}}}
		% \ecvitem{\rm \small (2017)}{\rm H. X. Chang, B. Qiao, Y. X. Zhang, Z. Xu, {\bf W. P. Yao}, C. T. Zhou and X. T. He}
%		\ecvitem{}{\rm This paper focuses on a new $\gamma$-ray emission mechanism from the interaction of ultraintense lasers and the near-critical-density plasmas. My contribution is primarily on the results analysis.}

  %   \ecvtitle{\color{black} {\it \small Phys. Rev. E}, {\bf 93}, {\rm 033206}}{\small \href{http://link.aps.org/doi/10.1103/PhysRevE.93.033206}{\underline{Characterization of magnetic reconnection in the high-energy-density regime}}}
		% \ecvitem{\rm \small (2016)}{\rm {Z. Xu}, B. Qiao, H. X. Chang, {\bf W. P. Yao}, S. Z. Wu, X. Q. Yan, C. T. Zhou, X. G. Wang and X. T. He}
%		\ecvitem{}{\rm The paper is motivated by laser driven MR experiments and focuses on the key results of MR from the diagnosed phenomena. My contribution is primarily on the results analysis.}

	 % \ecvtitle{\color{black} {\it \small Phys. Rev. A}, {\bf 98}, {\rm 052119}}{\small \href{https://journals.aps.org/pra/abstract/10.1103/PhysRevA.98.052119}{\underline{Identifying quantum radiation reaction by using colliding ultra-intense lasers in gases}}}
		% \ecvitem{\rm \small (2018)}{\rm X. B. Li, B. Qiao,  H. X. Chang,  H. He,  {\bf W. P. Yao},  X. F. Shen, J. Wang,  Y. Xie, C. L. Zhong, C. T. Zhou, S. P. Zhu,  and X. T. He}
%		\ecvitem{}{\rm This paper focuses on identifying quantum radiation reaction effect by using colliding ultra-intense lasers in a tenuous gas jet. My contribution is primarily on the results analysis.}

	 % \ecvtitle{\color{black} {\it \small Accepted in New J. Phys. }}{{\small High-flux high-energy ion beam production from stable collisionless shock acceleration by intense petawatt-picosecond laser pulses}}
		% \ecvitem{}{\rm H. He, B. Qiao, X. F. Shen, {\bf W. P. Yao}, Y. Xie, C. T. Zhou, X. T. He, S. P. Zhu, W. B. Pei and S. Z. Fu}
%		\ecvitem{}{\rm This paper focuses on stable collisonless shock ion acceleration in a high-Z solid tube via intense petawatt-picosecond laser pulses. My contribution is primarily on the results analysis.}
	
	 % \ecvtitle{\color{black} {\it \small Under review in Nat. Comm. }}{{\small Relativistic reconnection in near-Schwinger magnetic fields}}
		% \ecvitem{}{\rm Z. Xu, B. Qiao, {\bf W. P. Yao}, H. X. Chang, H. Zhang, F. L. Wang, C. T. Zhou, S. P. Zhu, G. Zhao, X. G. Wang and X. T. He}
%		\ecvitem{}{\rm The paper focuses on the QED effects on the relativistic reconnection in the extreme astrophysical environment. My contribution is primarily on the code testing and results analysis.}
	
	 % \ecvtitle{\color{black} {\it \small To be submitted }}{{\small Proton radiography for asymmetrical magnetic reconnection}}
		% \ecvitem{}{\rm Z. H. Zhao, B. Qiao, H. H. An, {\bf W. P. Yao}, W. Wang and X. T. He}
%		\ecvitem{}{\rm This paper focuses on using proton radiography to investigate asymmetrical MR in experiments. My contribution is primarily on the experiment design and results analysis.}	
%
%
%%\newpage
%
%\ecvsection{References}
%
%	\ecvtitle{\color{black} Name}{\color{black} Xiantu He}
%		\ecvitem{Title}{Professor of Physics; Academician of the Chinese Academy of Sciences;}
%		\ecvitem{Affiliation}{Center for Applied Physics and Technology, Peking University, Beijing}
%		\ecvitem{Email}{xthe@pku.edu.cn}
%		
%	\ecvtitle{\color{black} Name}{\color{black} Bin Qiao}
%		\ecvitem{Title}{Professor of Physics; Assistant Director, Center for Applied Physics and Technology}
%		\ecvitem{Affiliation}{School of Physics, Peking University, Beijing, China}
%		\ecvitem{Email}{bqiao@pku.edu.cn}
%	
%	\ecvtitle{\color{black} Name}{\color{black} Mingyang Yu}
%		\ecvitem{Title}{Professor of Physics}
%		\ecvitem{Affiliation}{Department of Physics, Zhejiang University, Hangzhou, China}
%		\ecvitem{Email}{myyu@zju.edu.cn}		
%
%	% \ecvtitle{\color{black} Name}{\color{black} Chikang Li}
%	% 	\ecvitem{Title}{Senior Research Scientist; Fellow of American Physical Society; Associate Head of High-Energy-Density Physics Division at MIT PSFC}
%	% 	\ecvitem{Affiliation}{Plasma Science and Fusion Center, Massachusetts Institute of Technology, Massachusetts, USA}
%	% 	\ecvitem{Email}{li@psfc.mit.edu}			
%
%\newpage
%	\ecvsection{Statement of Research Interests}
%{\rm	
%	\ecvitem{Background}{My PhD dissertation is devoted to the kinetic effects of astrophysical relativistic jet transport in plasmas, a branch of high-energy-density (HED) laboratory astrophysics. Astrophysical relativistic jets are associated with the most exciting high-energy astrophysical phenomena, such as Gamma-ray burst, active galactic nucleus, tidal disruption events and compact binary stars. Study of the relativistic jets transport can provide us with knowledge of their host sources and opportunities to advance our understanding of the Universe.}
%	
%	\ecvitem{}{However, no consensus has been reached concerning the basic parameters of relativistic jets. Although there are extensive observations via state-of-art space telescopes, more consistent explanations are still needed, especially in regions like the X-point of magnetic reconnection (jet lateral) and the discontinuity surface of collisionless shock (jet head).}
%	
%	\ecvitem{}{All the above issues call for kinetic treatment of relativistic jet transport, in which the fundamental interplay between charged particles and electromagnetic (EM) field is captured from first principle, such as the excitement, growth, competition and saturation of different kinetic instability modes, the generation of magnetic field, the formation of collisionless shock, the onset of magnetic reconnection, the energy dissipation and particle acceleration.}
%
%	\ecvitem{Current work:}{We have developed a novel injection module in our PIC code, which enables us to inject relativistic jets with various initial parameters including densities, drifting velocities, compositions, and distributions. With this platform, we systematically study the relativistic jet-ambient interaction and find many interesting results. For example,}
%	
%	\ecvitem{}{\begin{ecvitemize}
%		\item The jet component, or baryon loading effect (BLE) heavily affects relativistic jets transport dynamics in the interstellar medium. With the BLE, more free energy from baryons is provided to the Weibel instability at the jet body and the Buneman instability at the jet head. Thus, the jet electrons are able to draw energy from jet baryons, resulting in a stronger collisionless shock formation and a longer transport distance.
%		\item The external applied magnetic field largely shapes the spectrum energy distribution (SED) of jet electrons and the power law slope of the SED decreases as the magnetic field strength increases. When only jet electrons are fully magnetized, they are directly deflected by magnetic field, but jet protons are mainly dragged by charge-separation field. However, when both electrons and protons are fully magnetized, the contrary is the case.
%		\item The quantum electrodynamics (QED) cascade processes tremendously alter the relativistic jet transport around neutron stars. The transport of relativistic flows in extreme magnetic fields can be achieved in a relay manner, where both the quantum synchrotron radiation and pair creation are strongly coupled self-consistently. As photons play a key role as a medium, a new energy transport path is realized through photon radiation.
%		\item Besides, I have participated in other aspects of laboratory astrophysics, such as laser-driven magnetic reconnection and proton radiography experiment.
%		\item Moreover, familiar with the extreme conditions, I have also participated in the study of laser particle accelerations, as well as the radiations in QED-dominated regime.
%	\end{ecvitemize}}
%
%	\ecvitem{Future interest:}{With the development of laser technology, it will be more and more convenient to create plasmas under HED regime in laboratory. As it is extremely challenging to generate relativistic jets in laboratory for now, I am also interested in the non-relativistic jet transport/collimation, plasma instabilities saturation and collisonless/collisional shock formation and particle acceleration, especially when magnetic fields are externally applied. Additionally, I would like to participate more deeply in experiments on laser-driven magnetic field generation, turbulent amplification and magnetic reconnection, including the associated proton radiographic diagnoses. Besides, with a strong background in numerical simulation, I am interested in implementing new physical modules and advanced numerical schemes to simulations codes.}

	% \ecvitem{}{On the other hand, }

%	\ecvitem{}{\color{blue} In short, my research interest lies mainly in this new, physically rich and exciting area at the intersection of traditional plasma physics, high-energy astrophysics and laser-driven high-energy-density plasma physics.}




% }

	                 
  	\end{europasscv}
\end{document}